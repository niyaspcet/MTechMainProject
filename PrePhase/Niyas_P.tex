\documentclass{beamer}

\mode<presentation> {
%\usetheme{default}
%\usetheme{AnnArbor}
%\usetheme{Antibes}
%\usetheme{Bergen}
%\usetheme{Berkeley}
%\usetheme{Berlin}
%\usetheme{Boadilla}
%\usetheme{CambridgeUS}
%\usetheme{Copenhagen}
%\usetheme{Darmstadt}
%\usetheme{Dresden}
%\usetheme{Frankfurt}
%\usetheme{Goettingen}
%\usetheme{Hannover}
%\usetheme{Ilmenau}
%\usetheme{JuanLesPins}
%\usetheme{Luebeck}
\usetheme{Madrid}
%\usetheme{Malmoe}
%\usetheme{Marburg}
%\usetheme{Montpellier}
%\usetheme{PaloAlto}
%\usetheme{Pittsburgh}
%\usetheme{Rochester}
%\usetheme{Singapore}
%\usetheme{Szeged}
%\usetheme{Warsaw}

%\usecolortheme{albatross}
%\usecolortheme{beaver}
%\usecolortheme{beetle}
%\usecolortheme{crane}
%\usecolortheme{dolphin}
%\usecolortheme{dove}
%\usecolortheme{fly}
%\usecolortheme{lily}
%\usecolortheme{orchid}
%\usecolortheme{rose}
%\usecolortheme{seagull}
%\usecolortheme{seahorse}
%\usecolortheme{whale}
%\usecolortheme{wolverine}
%\setbeamertemplate{footline} % To remove the footer line in all slides uncomment this line
%\setbeamertemplate{footline}[page number] % To replace the footer line in all slides with a simple slide count uncomment this line

%\setbeamertemplate{navigation symbols}{} % To remove the navigation symbols from the bottom of all slides uncomment this line
}

\usepackage{graphicx} % Allows including images
\usepackage{booktabs} % Allows the use of \toprule, \midrule and \bottomrule in tables


\setbeamertemplate{headline}{%
\leavevmode%
  \hbox{%
    \begin{beamercolorbox}[wd=\paperwidth,ht=2.5ex,dp=1.125ex]{palette quaternary}%
    \insertsectionnavigationhorizontal{\paperwidth}{}{\hskip0pt plus1filll}
    \end{beamercolorbox}%
  }
}

%----------------------------------------------------------------------------------------
%	TITLE PAGE
%----------------------------------------------------------------------------------------

\title[Visual Aid for Blind]{Visual Aid for Blind} % The short title appears at the bottom of every slide, the full title is only on the title page

\author{Niyas P} % Your name

\institute[College of Engineering, Trivandrum] % Your institution as it will appear on the bottom of every slide, may be shorthand to save space
{TVE17ECSP10\\M. Tech (Signal Procesing)\\Third Semester\\
 % Your institution for the title page
\medskip
\textit{https://github.com/niyaspcet/MTechMainProject}\\ % Your email address
\vspace{1cm}
\textit{Guide}\\
Dr. Sreelatha G.\\
Department of ECE\\
College of Engineering, Trivandrum


}
\date{\today} % Date, can be changed to a custom date
\setbeamertemplate{bibliography item}[text]
\begin{document}

\begin{frame}[plain]

\titlepage % Print the title page as the first slide
\end{frame}

\begin{frame}
\frametitle{Overview} % Table of contents slide, comment this block out to remove it
\tableofcontents % Throughout your presentation, if you choose to use \section{} and \subsection{} commands, these will automatically be printed on this slide as an overview of your presentation
\end{frame}

%----------------------------------------------------------------------------------------
%	PRESENTATION SLIDES
%----------------------------------------------------------------------------------------

%------------------------------------------------
\section{Introduction} % Sections can be created in order to organize your presentation into discrete blocks, all sections and subsections are automatically printed in the table of contents as an overview of the talk
%------------------------------------------------


\begin{frame}
\frametitle{Introduction}
\begin{itemize}
\item
vision plays a vital role in gaining knowledge of the surrounding
world.
\item
According to the WHO
\begin{itemize}
\item
There are 285 million people in the world with visual
impairment.
\item
39 million of whom are blind
\end{itemize}
\item
Several systems were designed to improve the quality of life of VI (Visually Impaired) people.
\item
White cane and guide dogs were used traditionally
\item
Electronic aids are used nowadays
\item
ETA (Electronic Travel Aids) is an example of such sytem
\end{itemize}
\end{frame}

%------------------------------------------------
\section{Motivation}
\begin{frame}
\frametitle{Motivation}
\begin{itemize}
\item
90\% of VI people lives in developing countries like india
\item
Improving the quality of life of such people is one of the challenging task
\item
Even though there were different travel aids, the acceptance of visual aids are quite low among VI impaired people, which implies further researches.
\end{itemize}
\end{frame}

\section{Problem Statement}
\begin{frame}
\frametitle{}
\begin{block}{Problem Statement}
Develop a visual aid which helps visual impaired people with great acceptance among visually impaired people
\end{block}
\end{frame}

\section{Possible Solutions}
\begin{frame}
\frametitle{Possible Solutions}
Commonly using visual aids are
\begin{itemize}
\item
Navigation Aids
\item
Object and people identification aids
\item
Reading Aids
\item
Invasive techniques such
as implants
\item
Senory Substitution Devices (SSD)

\end{itemize}
\end{frame}




\section{Choosen Solution}
\begin{frame}
\frametitle{}
\begin{block}{Choosen Solution}
Visual Aid for visually impaired people by using non implant Sensory Substitution Device
\end{block}
\end{frame}


\begin{frame}
\frametitle{SSD}
\begin{itemize}
\item
Invasive techniques such
as implants provide low resolution imagery by stimulating surviving
retinal cells, cortex or optic nerve.
\begin{itemize}
\item
Risk associated with surgical procedures
\item
Expensive
\end{itemize}
\item{\textbf{SSD}:}
Non-invasive methods rely on human-centered computing
that bring together signal processing and person-centered computing
to harness the plasticity of the person's brain to process
information usually attributed to the impaired modality via an
unimpaired modality\cite{Brown2016}.
\begin{itemize}
\item
Visual to Tactile (VT)
\item
Visual to Auditory (VA)
\end{itemize}
\end{itemize}
\end{frame}
%------------------------------------------------


%------------------------------------------------
\section{Reference}
\begin{frame}[allowframebreaks]
\frametitle{References}
\bibliography{Bibl}
%\bibliographystyle{IEEETran}
%\scriptsize{\bibliographystyle{acm}}
\scriptsize{\bibliographystyle{IEEETran}}
\nocite{*}
\end{frame}

%------------------------------------------------
\section{Questions}
\begin{frame}
\Huge{\centerline{Questions ?}}
\end{frame}
\begin{frame}
\Huge{\centerline{Thank You}}
\end{frame}

%----------------------------------------------------------------------------------------

\end{document}